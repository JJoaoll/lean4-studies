% Created 2025-03-01 sáb 16:38
% Intended LaTeX compiler: pdflatex
\documentclass[11pt]{scrlttr2}
\usepackage[utf8]{inputenc}
\usepackage[T1]{fontenc}
\usepackage{graphicx}
\usepackage{longtable}
\usepackage{wrapfig}
\usepackage{rotating}
\usepackage[normalem]{ulem}
\usepackage{amsmath}
\usepackage{amssymb}
\usepackage{capt-of}
\usepackage{hyperref}
\setkomavar{fromname}{João Lucas de Moraes Pereira}
\setkomavar{fromemail}{draell@draell-nixos}
\KOMAoption{fromemail}{false}
\KOMAoption{fromphone}{false}
\KOMAoption{fromurl}{false}
\KOMAoption{fromlogo}{false}
\KOMAoption{backaddress}{false}
\setkomavar{place}{}
\KOMAoptions{foldmarks=true}
\LoadLetterOption{NF}
\date{\today}
\hypersetup{
 pdfauthor={João Lucas de Moraes Pereira},
 pdftitle={},
 pdfkeywords={},
 pdfsubject={},
 pdfcreator={Emacs 29.4 (Org mode 9.7.22)}, 
 pdflang={English}}
\begin{document}

\begin{letter}{%
\mbox{}}

\opening{Functional Programming in Lean}

lean4 does have a repl integrated with the LSP enviroment.
U should use the `\#eval` to evaluate, such as this example show us:
\begin{verbatim}
  #eval 1 + 2
\end{verbatim}

Also the function application works just like in Haskell.
(🔍) \textasciitilde{} So, i've just figured out that lean4 is a strict language, not a strict one. What a shame!
When necessary, u can use type coercions with ``:''
\begin{verbatim}
    #eval (1 + 2 : Nat)
    -- So look this:
    #eval  1 - 2        -->  0 : Nat
    #eval (1 - 2 : Int) --> -1 : Int
\end{verbatim}

As u can check the values with  `\#eval`
u also can check the types with `\#check`
\begin{verbatim}
    #check (1 - 2 : Int)  --> 1 - 2 : Int (without performing the operation.)
\end{verbatim}
to definitions, use `def`, like:
\begin{verbatim}
    def hello := "Hello"
\end{verbatim}

Note how great the `:=` fits for assignments instead of `=`.
If u wanna pick the type too, u can do it like this:
\begin{verbatim}
    def lean : String := "Lean"
\end{verbatim}

When it comes to define a function, there are a variaty of ways to do that.
The simpliest one (they say it):
\begin{verbatim}
    def succ (n : Nat) : Nat := n + 1
    -- or:
    def pred (n : Nat) : Nat :=
        n - 1
\end{verbatim}


Look this another one with more arguments:
\begin{verbatim}
    def maximum (n : Nat) (k : Nat) : Nat :=
      if n < k then
        k
      else n
\end{verbatim}

Lean4 does use currying as default.
(my exercises are in somewhere but it's not my intent to share them :P)
U use the same `def` keyword just like with the values (including functions)
\begin{verbatim}
    def Str  : Type := String
    def aStr : Str  := "Look at me, i'm a real String!"
\end{verbatim}

some explanation about ``aStr'' definition:
``The reason this works is that types follow the same rules as the rest of Lean. Types are expressions,
and in an expression, a defined name can be replaced with its definition. Because Str has been defined
to mean String, the definition of aStr makes sense.''

Something very very weird about Lean4 is that, because they want to overload number literals,
this code generates an error:
\begin{verbatim}
    def NaturalNumber : Type := Nat
    def thirtyEight : NaturalNumber := 38

    -- failed to synthesize instance
    -- OfNat NaturalNumber 38
\end{verbatim}

But it works this way:
\begin{verbatim}
    def thirtyEight : NaturalNumber := (38 : Nat)
\end{verbatim}

why?
"One way to work around this limitation is by providing the type Nat on the right-hand side of the definition, causing Nat's overloading rules to be used for 38:

The definition is still type-correct because NaturalNumber is the same type as Nat—by definition!"

But don't worry about it, we can see the light here:
``Another solution is to define an overloading for NaturalNumber that works equivalently to the one for Nat. This requires more advanced features of Lean, however.''

Finally we have some ``abbrev'' like here:
\begin{verbatim}
    abbrev N : Type := Nat
    def thirtyNine : N := 39
    -- works well
\end{verbatim}
whats a ``foldable | unfoldable | folded | unfolded'' definitions?
So they're basically product types..
\begin{verbatim}
    structure Point where
      x : Float
      y : Float
    deriving Repr

    def origin : Point := { x := 0.0, y := 0.0 }
\end{verbatim}

Note that the deriving works as the `Show` in haskell.
When u call the ``\#eval'', it call's the ``repr''. Look's like
we're gonna discover TypeClasses again..

About extracting things out of products we have:
\begin{verbatim}
    #eval origin.x
    #eval origin.y

    def addPoints (p1 : Point) (p2 : Point) : Point :=
        { x := p1.x + p2.x, y := p1.y + p2.y }


    structure Point3D where
        x : Float
        y : Float
        z : Float
    deriving Repr

    def origin3D : Point3D :=
        { x := 0.0, y := 0.0, z := 0.0 }
\end{verbatim}

``To make programs more concise, Lean also allows the structure type annotation inside the curly braces.''
\begin{verbatim}
    #check { x := 0.0, y := 0.0 : Point}
\end{verbatim}

Another convenient syntax is:
\begin{verbatim}
    -- default
    def zeroX (p : Point) : Point :=
        { x := 0, y := p.y }
    -- with syntax
    def zeroX (p : Point) : Point :=
        { p with x := 0 }
\end{verbatim}

With this syntax, some issues about refactoring code, e.g., adding new
new columns to a struct would not implie that u're refactoring every
single function that does things like above.
So.. In Lean4, ure forced to use the ``mk'' thing that
u could avoid in Haskell after get some maturity..
\begin{verbatim}
    #check Point.mk 1.5 2.8
      --> { x := 1.5, y := 2.8 } : Point
\end{verbatim}
It's just exacly how a constructor should be in a product type,
but forcing u to use de `(\textsubscript{.mk})` syntax, and what's ok.

\begin{itemize}
\item Fine, i was wrong. And the default.. Nice one, lean, nice one.
\end{itemize}
look at that:
\begin{verbatim}
    structure Point where
        point ::
        x : Float
        y : Float
    deriving Repr
\end{verbatim}

Some notes about 'behind-the-scenes':
``In fact, just as the curly-braced structure construction syntax is converted to a call to the structure's constructor behind the scenes, the syntax p1.x in the prior definition of addPoints is converted into a call to the Point.x accessor. That is, \#eval origin.x and \#eval Point.x origin both yield''

LOOK AT THIS!!:
    Accessor dot notation is usable with more than just structure fields. It can also be used for functions that take any number of arguments. More generally, accessor notation has the form TARGET.f ARG1 ARG2 \ldots{}. If TARGET has type T, the function named T.f is called. TARGET becomes its leftmost argument of type T, which is often but not always the first one, and ARG1 ARG2 \ldots{} are provided in order as the remaining arguments. For instance, String.append can be invoked from a string with accessor notation, even though String is not a structure with an append field.

\begin{verbatim}
    #eval "one string".append " and another"
\end{verbatim}

(question?) - how the parenthesis works with `\#eval` and `\#check`?
It's module starts presenting what's a sum and a product of types in a
very simple and didactic way.

Also, the concept ``inductive data types'' were explained in this short-text:
``Datatypes that allow choices are called sum types and datatypes that can include instances of themselves are called recursive datatypes. Recursive sum types are called inductive datatypes, because mathematical induction may be used to prove statements about them. When programming, inductive datatypes are consumed through pattern matching and recursive functions.''

But also look this definition:
\#+begin\textsubscript{src} lean4
    inductive Bool where
\begin{center}
\begin{tabular}{l}
false : Bool\\
true : Bool\\
\end{tabular}
\end{center}
+end\textsubscript{src}
\closing{}




\end{letter}

\end{document}
